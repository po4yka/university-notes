% В этом документе преамбула

\documentclass[a4paper,12pt]{article}

\usepackage{lscape} % горизонтальный режим
\usepackage{pdflscape}

\usepackage{lipsum} % тестовые тексты

%%% Работа с русским языком
\usepackage{cmap}					% поиск в PDF
\usepackage{mathtext} 				% русские буквы в формулах
\usepackage[T2A]{fontenc}			% кодировка
\usepackage[utf8]{inputenc}			% кодировка исходного текста
\usepackage[english,russian]{babel}	% локализация и переносы
\usepackage{indentfirst}			% чтобы первый абзац в разделе отбивался красной строкой
\frenchspacing						% тонкая настройка пробелов
\usepackage{gensymb}				% символы по типу градусов
\usepackage{algorithm}				% для алгоритмов
\usepackage{algorithmic}			% для алгоритмов

%%% Приведение начертания букв и знаков к русской типографской традиции
\renewcommand{\epsilon}{\ensuremath{\varepsilon}}
\renewcommand{\phi}{\ensuremath{\varphi}}			% буквы "эпсилон"
\renewcommand{\kappa}{\ensuremath{\varkappa}}		% буквы "каппа"
\renewcommand{\le}{\ensuremath{\leqslant}}			% знак меньше или равно
\renewcommand{\leq}{\ensuremath{\leqslant}}			% знак меньше или равно
\renewcommand{\ge}{\ensuremath{\geqslant}}			% знак больше или равно
\renewcommand{\geq}{\ensuremath{\geqslant}}			% знак больше или равно
\renewcommand{\emptyset}{\varnothing}				% знак пустого множества

%%% Дополнительная работа с математикой
\usepackage{amsmath,amsfonts,amssymb,amsthm,mathtools,esint} % AMS
\usepackage{wasysym}
\usepackage{icomma} % "Умная" запятая: $0,2$ --- число, $0, 2$ --- перечисление

%% Номера формул
\mathtoolsset{showonlyrefs=true} % Показывать номера только у тех формул, на которые есть \eqref{} в тексте.

%% Перенос знаков в формулах (по Львовскому)
\newcommand*{\hm}[1]{#1\nobreak\discretionary{}
{\hbox{$\mathsurround=0pt #1$}}{}}

%%% Работа с картинками
\usepackage{graphicx}  				% Для вставки рисунков
\graphicspath{{images/}{images2/}}  % папки с картинками
\setlength\fboxsep{3pt} 			% Отступ рамки \fbox{} от рисунка
\setlength\fboxrule{1pt} 			% Толщина линий рамки \fbox{}
\usepackage{wrapfig} 				% Обтекание рисунков текстом

%%% Работа с таблицами
\usepackage{array,tabularx,tabulary,booktabs} 	% Дополнительная работа с таблицами
\usepackage{longtable}  						% Длинные таблицы
\usepackage{multirow}							% Слияние строк в таблице

%----------------------------------------------------------------------------------------------

%%% НАСТРОЙКА СТИЛЕЙ

\newtheoremstyle{break}% name
	{}%         Space above, empty = `usual value'
	{}%         Space below
	{\itshape}% Body font
	{}%         Indent amount (empty = no indent, \parindent = para indent)
	{\bfseries}% Thm head font
	{.}%        Punctuation after thm head
	{\newline}% Space after thm head: \newline = linebreak
	{}%         Thm head spec
\theoremstyle{break}

\newtheorem{lemma}{Лемма}[section]

\newtheoremstyle{mydef}
	{\topsep}%
	{\topsep}%
	{}%
	{}%
	{\bfseries}%
	{}%
	{\newline}%
	{%
		\rule{\textwidth}{0.4pt}\\*%
		\thmname{#1}~\thmnumber{#2}\thmnote{\ -\ #3}.\\*[-1.5ex]%
		\rule{\textwidth}{0.4pt}}%

\theoremstyle{mydef}
\newtheorem{definition}{Определение}[section]

\newtheorem{property}{Свойство}

\newtheoremstyle{inline}%
	{3pt}%
	{3pt}%
	{\normalfont}%
	{}%
	{\bfseries}%
	{}%
	{\newline}%
	{\thmname{#1}\thmnumber{ #2}\thmnote{ (#3)}}%
\theoremstyle{inline}
\newtheorem{exmp}[definition]{Пример}
\newtheorem{theorem}{Теорема}[section]
\newtheorem{axioms}{Аксиомы}[section]
\newtheorem{corollary}{Следствие}[theorem]

\theoremstyle{remark} % "Примечание"
\newtheorem*{nonum}{Решение}
\newtheorem*{remark}{Примечание}

%----------------------------------------------------------------------------------------------

%%% Страница
\usepackage{extsizes} % Возможность сделать 14-й шрифт
\usepackage{geometry} % Простой способ задавать поля
	\geometry{top=15mm}
	\geometry{bottom=35mm}
	\geometry{left=10mm}
	\geometry{right=10mm}

%\usepackage{fancyhdr} % Колонтитулы
% 	\pagestyle{fancy}
 	%\renewcommand{\headrulewidth}{0pt}  % Толщина линейки, отчеркивающей верхний колонтитул
% 	\lfoot{Нижний левый}
% 	\rfoot{Нижний правый}
% 	\rhead{Верхний правый}
% 	\chead{Верхний в центре}
% 	\lhead{Верхний левый}
%	\cfoot{Нижний в центре} % По умолчанию здесь номер страницы

\usepackage{setspace} % Интерлиньяж (межстрочные интервалы)
%\onehalfspacing % Интерлиньяж 1.5
%\doublespacing % Интерлиньяж 2
%\singlespacing % Интерлиньяж 1

\usepackage{lastpage} % Узнать, сколько всего страниц в документе.

\usepackage{soulutf8} % Модификаторы начертания

\usepackage{hyperref}
\usepackage[usenames,dvipsnames,svgnames,table,rgb]{xcolor}
\hypersetup{				% Гиперссылки
    unicode=true,           % русские буквы в раздела PDF
    pdftitle={Заголовок},   % Заголовок
    pdfauthor={Автор},      % Автор
    pdfsubject={Тема},      % Тема
    pdfcreator={Создатель}, % Создатель
    pdfproducer={Производитель}, % Производитель
    pdfkeywords={keyword1} {key2} {key3}, % Ключевые слова
    colorlinks=true,       	% false: ссылки в рамках; true: цветные ссылки
    linkcolor=MidnightBlue, % внутренние ссылки
    citecolor=black,        % на библиографию
    filecolor=magenta,      % на файлы
    urlcolor=blue           % на URL
}

\usepackage{csquotes} % Еще инструменты для ссылок

%\usepackage[style=authoryear,maxcitenames=2,backend=biber,sorting=nty]{biblatex}

\usepackage{multicol} % Несколько колонок

%%% Работа с графикой
\usepackage{tikz}
\usetikzlibrary{calc}
\usepackage{tkz-euclide}
\usetikzlibrary{arrows}
\usepackage{pgfplots}
\usepackage{pgfplotstable}

%%% Настройка подписей к плавающим объектам
% \usepackage{floatrow}	% размещение
\usepackage{caption}	% начертание
\captionsetup[figure]{labelfont=bf,textfont=it,font=footnotesize}	% нумерация и надпись курсивом
% для подфигур: заголовок подписи полужирный, текст заголовка обычный
% выравнивание является неровным (т.е. выровненным по левому краю)
% singlelinecheck = off означает, что настройка выравнивания используется, даже если заголовок имеет длину только одну строку.
% если singlelinecheck = on, то заголовок всегда центрируется, когда заголовок состоит только из одной строки.
\captionsetup[subfigure]{labelfont=bf,textfont=normalfont,singlelinecheck=off,justification=raggedright}

%%% Stuff для листинга
\usepackage{listings}
\usepackage{xcolor}

\usepackage{subcaption}		% размещение рисунков
\usepackage{graphicx}

\colorlet{mygray}{black!30}
\colorlet{mygreen}{green!60!blue}
\colorlet{mymauve}{red!60!blue}

\lstset{
	backgroundcolor=\color{gray!10},  
	basicstyle=\ttfamily,
	columns=fullflexible,
	breakatwhitespace=false,      
	breaklines=true,                
	captionpos=b,                    
	commentstyle=\color{mygreen}, 
	extendedchars=true,              
	frame=single,                   
	keepspaces=true,             
	keywordstyle=\color{blue},      
	language=c++,                 
	numbers=none,                
	numbersep=5pt,                   
	numberstyle=\tiny\color{blue}, 
	rulecolor=\color{mygray},        
	showspaces=false,               
	showtabs=false,                 
	stepnumber=5,                  
	stringstyle=\color{mymauve},    
	tabsize=3,                      
	title=\lstname                
}

% Для извращённых начертаний
\usepackage{mathrsfs}

\usepackage{makecell}
\setcellgapes{3pt}

% Зачёркивание символов
\usepackage{cancel}

% перечисления с буквами
\usepackage{enumitem}

% Римские цифры
\newcommand{\RNumb}[1]{\uppercase\expandafter{\romannumeral #1\relax}}

% Дроби с косой чертой
\usepackage{nicefrac}

% Разделение ячеек таблицы
\usepackage{diagbox}

%----------------------------------------------------------------------------------------------

%% ОБОЗНАЧЕНИЯ ДЛЯ ТВИМС %%

\DeclareMathOperator{\sgn}{\mathop{sgn}}				% ф-ия sgn
\renewcommand{\tg}{\mathop{\mathrm{tg}}\nolimits}		% обозначение тангенса
\DeclareMathOperator{\cov}{\mathop{cov}}				% ковариация
\DeclareMathOperator{\corr}{\mathop{corr}}				% матрица кореляции
\DeclareMathOperator{\var}{\mathop{var}}				% матрица ковариации

\DeclareMathOperator\supp{supp}							% Носитель функции распределения
\DeclareMathOperator{\eq}{\Leftrightarrow}				% Знак эквивалентно

% Борелевская сигма-алгебра
\newcommand{\Borel}{{\mathcal{B}}}
% Мн-во случайных событий
\newcommand{\Event}{{\mathcal F}}

% Кольца чисел N,Z,R,C
\newcommand\NN{{\mathbb N}}
\newcommand\ZZ{{\mathbb Z}}
\newcommand\RR{{\mathbb R}}
\newcommand\CC{{\mathbb C}}

% Спецсимвол "Пусть"
\def\letus{%
	\mathord{\setbox0=\hbox{$\exists$}%
		\hbox{\kern 0.125\wd0%
			\vbox to \ht0{%
				\hrule width 0.75\wd0%
				\vfill%
				\hrule width 0.75\wd0}%
			\vrule height \ht0%
			\kern 0.125\wd0}%
	}%
}

\begin{document} % начало документа
	
\textit{Почаев Никита Алексеевич, гр. 8381}

\section{Комбинаторика, классическое определение вероятности. (ДЗ на 15.02.20)}

\subsection{Задача 1.}

2 шара распределены случайно по 3-м ящикам. Определить вероятность попадания в разные ящики.

\textit{Решение:}

Пусть $n$ - количество шаров, $k$ - количество ящиков. Тогда способов разместить шары по ящикам (в одном может быть не один): $k^n$ (первый можно положить в любой из $n$, второй также в любой из $n$ и т.д.). Значит, всего исходов: $3^2=9$.

Рассмотрим количество способов разместить шары без повторений, т.е. каждый шар по одному в разных ящиках. Первый шар можно положить в любой из $k$ ящиков, второй - в любой из оставшихся $k-1$ ящиков, третий - в один из оставшихся $k-2$ ящиков, $\dots$, последний - в любой из оставшихся $k-n+1$. Поэтому: $\dfrac{k!}{(k-n)!}$. Т.е. количество благоприятных исходов: $\dfrac{3!}{(3-2)!} = \dfrac{6}{1} = 6$.

Воспользуемся классическим определением вероятности: $P = \dfrac{m}{n}$, где $m$ - число исходов, благоприятствующих осуществлению события, а $n$ - число всех равновозможных элементарных исходов.

\noindent Ответ: $\dfrac{6}{9} = \dfrac{2}{3}$

\textit{Решение через математическую модель:}

\noindent Всего исходов: $\Omega = \{ \omega_{ij} \}, 1 \le i, j \le 3$, где $i, j$ - номера ящиков, куда попали соотвественно 1-ый и 2-ой шары.
\[ \# \Omega = 3^2 = 9 \]
\noindent Событие $A$ - $\{ \omega_{ij}, i \ne j \}$ - упорядоченный набор без повторений.
\[ \#A = A_3^2 = 3! = 6 \]
\[ P(A) = \dfrac{\# A}{\# \Omega} = \dfrac{6}{9} = \dfrac{2}{3} \]

\subsection{Задача 2.}

В колоде содержится 52 карты (полная колода, от двоек до тузов), наугад достают 5. Определить вероятности, что:
\begin{enumerate}
	\item[а)] все одной масти
	\item[б)] 3 одного достоинства, 2 - другого, 3+2
	\item[в)] 2 - одного достоинства, 2 другого, и 1 - третьего
	\item[г)] всё достоинства подряд (2, 3, 4, 5, 6)
\end{enumerate}

\textit{Решение:}

\begin{remark}
	Всего у нас 4 масти и по 13 карт в каждой. По 4 карты одного достоинства (13 достоинств).
\end{remark}

Всего способов извлечь 5 карт из колоды: $C_n^k = \dfrac{n!}{k!(n-k)!} = \dfrac{52!}{5!(52-5)!} = \dfrac{52 \cdot 51 \cdot 50 \cdot 49 \cdot 48 \cdot 47!}{5! \cdot 47!} = 2598960$ - это общее число исходов.

\begin{enumerate}
	\item Число различных мастей равно 4. Число карт одной масти равно 13. Поэтому число благоприятных исходов равно $4 \cdot C_{13}^5 = 4 \cdot 1287 = 5148$. $P = \dfrac{5148}{2598960} = \dfrac{33}{16660}$.
	
	Построим математическую модель. Все масти сходятся - событие $A$. Разделим на 4 события:
	\begin{itemize}
		\item[а)] $B: i, j, l, m \in [1;13]$
		\item[б)] $C: i, j, l, m \in [14;26]$
		\item[в)] $D: i, j, l, m \in [27;39]$
		\item[г)] $E: i, j, l, m \in [39;52]$
	\end{itemize}
	\noindent $\#B = \#C = \#D = \#E = C_{13}^5$
	\noindent $\#A = \#B + \#C + \#D + \#E = 4 \cdot C_{13}^5$.
	\item Построим математическую модель. Пусть $i,j,k$ - карты одного достоинства, а $l,m$ - другого.
	
	\noindent $\#A: (i,j,k \text{ одного достоинства}) = C_{13}^1 \cdot C_4^3 = 13 \cdot 4$ (выбираем сначала одно из 13 достоинств, в котором 4 карты, из которых выбираем 3).
	
	\noindent $\#B: (i,j,k \text{ одного достоинства и } l,m \text{ другого}) = \#A \cdot C_{12}^1 \cdot C_4^2 = 12 \cdot 6 \cdot 13 \cdot 4$
	
	\noindent $P(A) = \dfrac{C_{13}^1 \cdot C_4^3 \cdot C_{12}^1 \cdot C_4^2}{C_{52}^5}$
	\item Построим математическую модель. Пусть $i, j$ - одного достоинства, $k, l$ - другого, $m$ - третьего.
	
	\noindent $\#A: (i, j, \text{ одного достоинства}) = C_{13}^1 \cdot C_4^2 = 13 \cdot 6$
	
	\noindent $\#B: (k, l, \text{ другого достоинства}) = C_{12}^1 \cdot C_4^2 = 12 \cdot 6$
	
	\noindent $\#C: (m \text{ третьего}) = C_{11}^1 \cdot C_4^1 = 11 \cdot 4$
	
	\noindent $\#D(A, B, C) = 13 \cdot 6 \cdot 12 \cdot 6 \cdot 11 \cdot 4$
	
	\noindent $P(D) = \dfrac{\#D}{\# \Omega} = \dfrac{13 \cdot 6 \cdot 12 \cdot 6 \cdot 11 \cdot 4}{C_{52}^5}$
	\item Одну двойку (тройку, четвёрку, пятёрку, шестёрку) из 4 можно извлечь $ C_{4}^1$ способами. Поэтому: $P = \dfrac{\left(C_{4}^1\right)^5}{C_{52}^5} = \dfrac{1024}{2598960} = \dfrac{13}{33320}$. Однако, мы рассмотрели случай, когда эти карты просто содержатся в 5-ке, которую достали. Если нам необходима последовательность, т.е. мы достаём карты одна за другой, вероятность будем иной. $P = \dfrac{4}{52} \cdot \dfrac{4}{51} \cdot \dfrac{4}{50} \cdot \dfrac{4}{49} \cdot \dfrac{4}{48} = \dfrac{8}{2436525}$.
	
	\textit{Решим через математическую модель}, когда требуется просто все достоинства подряд в 5-ке (не конкретную комбинацию).
	
	\noindent Пусть $i, j, k, l, m$ - все достоинства подряд. $A: (i \text{ подходит для стрита}): 2 \le i \le 10$. 
	
	\noindent $\#A = 9 \cdot 4 = 36$.
	
	\noindent $B: \begin{pmatrix} i \\ j = i + 1 \\ k = j + 1 \\ l = k + 1 \\ m = l + 1 \end{pmatrix}$.
	
	\noindent Пусть $G(X)$ - достоинство $X$, тогда условие $G(i) = G(j) - 1 = G(k) - 2 = G(l) - 3 = G(m) - 4$ возможно при: $G(i) = G(m) - 4 > 0 \Rightarrow 5 \le G(m) \le 13 \Rightarrow \#B = 9$ - среди достоинства выбрано 5 идущих подряд.
	
	\noindent Т.к. мастей 4, то для каждой карты $\exists$ 4 варианта выбрать масть: $G(i) = G(i + 13) = G(i + 26) = G(i + 39)$, а т.к. при этом выбирается 5 карт, то: $\#B = \#A \cdot 4^4 = 36 \cdot 256 = 9216$.
	
	\noindent $P(B) = \dfrac{9216}{2598960}$
	
	P.S. Из предыдущего решения данное можно получить путём домножения на 9 - столько раз мы можем сдвигать подряд идущие карты от последней - шестёрки, до последней - туза.
\end{enumerate}

\subsection{Задача 3.}

Имеется 6 ящиков различных материалов и 5 этажей. Определить вероятность, что на 3 этаже имеется хотя бы один ящик.

\textit{Решение:}

\noindent Общая методика для решения задач, в которых встречается фраза <<хотя бы один>> такая:

\begin{enumerate}
	\item Выписать исходное событие $A$ = (Вероятность того, что ... хотя бы ...).
	\item Сформулировать противоположное событие $\bar A$.
	\item Найти вероятность события $P(\bar A)$.
	\item Найти искомую вероятность по формуле $P(A) = 1 - P(\bar A)$.
\end{enumerate}

\noindent $\bar A$ - не содержится ни одного ящика. Всего способов разместить 6 ящиков по 5 этажам: $\#\Omega = 5^6 = 15625$. Способов разместить ящики по 4 этажам, минуя 3-ий: $\# \bar A = 4^6$. В результате: $P(\bar A) = \dfrac{4096}{15625}$. Ответ: $P(A) = 1 - \dfrac{4096}{15625} = \dfrac{11529}{15625}$.

Для построения мат. модели необходимо обговорить, что $\{\omega_{i j k l m}\}$, где $i$ - этаж 1-го ящ., $j$ - этаж 2-го ящ. и т.д. Тогда $\# \bar A = \{\omega_{i j k l m}\}$, при $i,j,k,l,m \ne 3$.

\subsection{Задача 4.}

Код - три кнопки, нажатых, одновременно, 0-9. Код неизвестен. Определить вероятность, что замок открыт:
\begin{enumerate}
	\item[а)] код не запоминается
	\item[б)] код запоминается
\end{enumerate}

\begin{enumerate}
	\item на $k$-м шаге
	\item до $k$-ого шага
\end{enumerate}

\textit{Решение:}

Всего исходов: $\{w_{ijk}\}$, где $0 \le i,j,k \le 9$. Порядок $i,j,k$ не важен (для определённости положим, что $i \le j \le k$). Тогда число возможных исходов - сочетание из 10 по 3 с повторениями (т.к. могут быть одинаковые цифры, но порядок неважен):

\[ \# \Omega = C_{(n)}^k = \left( \begin{pmatrix} n \\ k \end{pmatrix} \right) = \begin{pmatrix} (n+k-1) \\ k \end{pmatrix} = (-1)^k\begin{pmatrix} -n \\ k \end{pmatrix} = \dfrac{(n+k-1)!}{k! \cdot (n-1)!} = C_{10+3-1}^{3} = C_{12}^3 = 220 \]
\begin{enumerate}
	\item[а)] Код не запоминается. 
	Пусть $A$ – событие, при котором на $k$-том шаге замок откроется. Т.к. код не запоминается, то на любом шаге вероятность открыть замок: $P(A) = \dfrac{1}{\# \Omega} = \dfrac{1}{220}$.
	
	\noindent Пусть $B_k$ – вероятность, что до $k$-то шага (включительно) замок откроется хотя бы один раз. Тогда ($\bar B_k$) – вероятность, что ни на одном шаге замок не откроется, то есть на каждом шаге случится событие $\bar A$.
	\[ P(\bar A) = 1 - P(A) = \dfrac{219}{220} \]
	\[ P(\bar B_k) = \prod_{i=1}^{k} P(\bar A) = \left( \dfrac{219}{220} \right)^k \]
	\[ P(B_k) = 1 - P(\bar B_k) = 1 - \left( \dfrac{219}{220} \right)^k \]
	
	\item[б)] Код запоминается.
	Пусть $A_k$ – вероятность, что на $k$-том шаге замок откроется. Т.к. код запоминается, то с каждым шагом отсекается один из вариантов, то есть выбирается один из $\# \Omega - k + 1$ возможных вариантов. Таким образом,
	\[ P(A_k) = \dfrac{1}{\# \Omega - k + 1} = \dfrac{1}{221 - k} \]
	
	\noindent Пусть $B_k$ – вероятность, что до $k$-то шага (включительно) замок откроется (хоть раз). Тогда $B_k$ - вероятность, что ни на одном шаге замок не откроется, то есть на каждом шаге случится событие $\bar A_k$.
	\[ P(\bar A_k) = A - P(A_k) = \dfrac{221 - k - 1}{221 - k} \]
	\[ P(\bar B_k) = \prod_{i=1}^{k} P(\bar A) = \prod_{i=1}^{k} \dfrac{221 - i - 1}{221 - i} = \dfrac{221 - k}{220} \]
	\[ P(B_k) = 1 - P(\bar B_k) = 1 - \dfrac{221 - k}{220} = \dfrac{k}{220} \]
\end{enumerate}

\end{document} % конец документа