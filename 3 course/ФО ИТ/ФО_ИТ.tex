% В этом документе преамбула

\documentclass[a4paper,12pt]{article}

\usepackage{lscape} % горизонтальный режим
\usepackage{pdflscape}

\usepackage{lipsum} % тестовые тексты

%%% Работа с русским языком
\usepackage{cmap}					% поиск в PDF
\usepackage{mathtext} 				% русские буквы в формулах
\usepackage[T2A]{fontenc}			% кодировка
\usepackage[utf8]{inputenc}			% кодировка исходного текста
\usepackage[english,russian]{babel}	% локализация и переносы
\usepackage{indentfirst}			% чтобы первый абзац в разделе отбивался красной строкой
\frenchspacing						% тонкая настройка пробелов
\usepackage{gensymb}				% символы по типу градусов
\usepackage{algorithm}				% для алгоритмов
\usepackage{algorithmic}			% для алгоритмов

%%% Приведение начертания букв и знаков к русской типографской традиции
\renewcommand{\epsilon}{\ensuremath{\varepsilon}}
\renewcommand{\phi}{\ensuremath{\varphi}}			% буквы "эпсилон"
\renewcommand{\kappa}{\ensuremath{\varkappa}}		% буквы "каппа"
\renewcommand{\le}{\ensuremath{\leqslant}}			% знак меньше или равно
\renewcommand{\leq}{\ensuremath{\leqslant}}			% знак меньше или равно
\renewcommand{\ge}{\ensuremath{\geqslant}}			% знак больше или равно
\renewcommand{\geq}{\ensuremath{\geqslant}}			% знак больше или равно
\renewcommand{\emptyset}{\varnothing}				% знак пустого множества

%%% Дополнительная работа с математикой
\usepackage{amsmath,amsfonts,amssymb,amsthm,mathtools,esint} % AMS
\usepackage{wasysym}
\usepackage{icomma} % "Умная" запятая: $0,2$ --- число, $0, 2$ --- перечисление

%% Номера формул
\mathtoolsset{showonlyrefs=true} % Показывать номера только у тех формул, на которые есть \eqref{} в тексте.

%% Свои команды

% операции, не определённые (или имеющие иные обохначения) в мат. пакетах
\DeclareMathOperator{\sgn}{\mathop{sgn}}				% ф-ия sgn
\renewcommand{\tg}{\mathop{\mathrm{tg}}\nolimits}		% обозначение тангенса

%% Перенос знаков в формулах (по Львовскому)
\newcommand*{\hm}[1]{#1\nobreak\discretionary{}
{\hbox{$\mathsurround=0pt #1$}}{}}

%%% Работа с картинками
\usepackage{graphicx}  				% Для вставки рисунков
\graphicspath{{images/}{images2/}}  % папки с картинками
\setlength\fboxsep{3pt} 			% Отступ рамки \fbox{} от рисунка
\setlength\fboxrule{1pt} 			% Толщина линий рамки \fbox{}
\usepackage{wrapfig} 				% Обтекание рисунков текстом

%%% Работа с таблицами
\usepackage{array,tabularx,tabulary,booktabs} 	% Дополнительная работа с таблицами
\usepackage{longtable}  						% Длинные таблицы
\usepackage{multirow}							% Слияние строк в таблице

%%% Теоремы

\newtheoremstyle{break}% name
	{}%         Space above, empty = `usual value'
	{}%         Space below
	{\itshape}% Body font
	{}%         Indent amount (empty = no indent, \parindent = para indent)
	{\bfseries}% Thm head font
	{.}%        Punctuation after thm head
	{\newline}% Space after thm head: \newline = linebreak
	{}%         Thm head spec
\theoremstyle{break}

% \theoremstyle{plain} % Стиль по умолчанию
\newtheorem{theorem}{Теорема}[section]
\newtheorem{lemma}{Лемма}[section]
\newtheorem{definition}[theorem]{Определение}
\newtheorem{property}{Свойство}
 
\newtheorem{corollary}{Следствие}[theorem]

\newtheoremstyle{example}	% style name
	{2ex}					% above space
	{2ex}					% below space
	{}						% body font
	{}						% indent amount
	{\bf}				% head font
	{.}						% post head punctuation
	{\newline}				% post head punctuation
	{\thmname{#1}\thmnumber{ #2}\thmnote{ (#3)}}						% head spec

\theoremstyle{example}
\newtheorem{exmp}{Пример}[section]
 
\theoremstyle{remark} % "Примечание"
\newtheorem*{nonum}{Решение}
\newtheorem*{evidence}{Доказательство}
\newtheorem*{remark}{Примечание}

%%% Программирование
\usepackage{etoolbox} % логические операторы

%%% Страница
\usepackage{extsizes} % Возможность сделать 14-й шрифт
\usepackage{geometry} % Простой способ задавать поля
	\geometry{top=15mm}
	\geometry{bottom=35mm}
	\geometry{left=10mm}
	\geometry{right=10mm}

%\usepackage{fancyhdr} % Колонтитулы
% 	\pagestyle{fancy}
 	%\renewcommand{\headrulewidth}{0pt}  % Толщина линейки, отчеркивающей верхний колонтитул
% 	\lfoot{Нижний левый}
% 	\rfoot{Нижний правый}
% 	\rhead{Верхний правый}
% 	\chead{Верхний в центре}
% 	\lhead{Верхний левый}
%	\cfoot{Нижний в центре} % По умолчанию здесь номер страницы

\usepackage{setspace} % Интерлиньяж (межстрочные интервалы)
%\onehalfspacing % Интерлиньяж 1.5
%\doublespacing % Интерлиньяж 2
%\singlespacing % Интерлиньяж 1

\usepackage{lastpage} % Узнать, сколько всего страниц в документе.

\usepackage{soulutf8} % Модификаторы начертания

\usepackage{hyperref}
\usepackage[usenames,dvipsnames,svgnames,table,rgb]{xcolor}
\hypersetup{				% Гиперссылки
    unicode=true,           % русские буквы в раздела PDF
    pdftitle={Заголовок},   % Заголовок
    pdfauthor={Автор},      % Автор
    pdfsubject={Тема},      % Тема
    pdfcreator={Создатель}, % Создатель
    pdfproducer={Производитель}, % Производитель
    pdfkeywords={keyword1} {key2} {key3}, % Ключевые слова
    colorlinks=true,       	% false: ссылки в рамках; true: цветные ссылки
    linkcolor=MidnightBlue, % внутренние ссылки
    citecolor=black,        % на библиографию
    filecolor=magenta,      % на файлы
    urlcolor=blue           % на URL
}

\usepackage{csquotes} % Еще инструменты для ссылок

%\usepackage[style=authoryear,maxcitenames=2,backend=biber,sorting=nty]{biblatex}

\usepackage{multicol} % Несколько колонок

%%% Работа с графикой
\usepackage{tikz}
\usetikzlibrary{calc}
\usepackage{tkz-euclide}
\usetikzlibrary{arrows}
\usepackage{pgfplots}
\usepackage{pgfplotstable}

%%% Настройка подписей к плавающим объектам
% \usepackage{floatrow}	% размещение
\usepackage{caption}	% начертание
\captionsetup[figure]{labelfont=bf,textfont=it,font=footnotesize}	% нумерация и надпись курсивом
% для подфигур: заголовок подписи полужирный, текст заголовка обычный
% выравнивание является неровным (т.е. выровненным по левому краю)
% singlelinecheck = off означает, что настройка выравнивания используется, даже если заголовок имеет длину только одну строку.
% если singlelinecheck = on, то заголовок всегда центрируется, когда заголовок состоит только из одной строки.
\captionsetup[subfigure]{labelfont=bf,textfont=normalfont,singlelinecheck=off,justification=raggedright}

%%% Stuff для листинга
\usepackage{listings}
\usepackage{xcolor}

\colorlet{mygray}{black!30}
\colorlet{mygreen}{green!60!blue}
\colorlet{mymauve}{red!60!blue}

\lstset{
	backgroundcolor=\color{gray!10},  
	basicstyle=\ttfamily,
	columns=fullflexible,
	breakatwhitespace=false,      
	breaklines=true,                
	captionpos=b,                    
	commentstyle=\color{mygreen}, 
	extendedchars=true,              
	frame=single,                   
	keepspaces=true,             
	keywordstyle=\color{blue},      
	language=c++,                 
	numbers=none,                
	numbersep=5pt,                   
	numberstyle=\tiny\color{blue}, 
	rulecolor=\color{mygray},        
	showspaces=false,               
	showtabs=false,                 
	stepnumber=5,                  
	stringstyle=\color{mymauve},    
	tabsize=3,                      
	title=\lstname                
}

% для извращённых начертаний
\usepackage{mathrsfs}

\usepackage{makecell}
\setcellgapes{3pt}

% Зачёркивание символов
\usepackage{cancel}

% перечисления с буквами
\usepackage{enumitem}

\DeclareMathOperator{\rot}{\text{rot}}
\DeclareMathOperator{\divFiz}{\text{div}}
\DeclareMathOperator{\grad}{\text{grad}}

\begin{document} % начало документа

\section*{Темы для повторения}

Уравнение Максвелла, дифференциальные уравнения, уравнения Д'Аламбера, полупроводниковые устройства (от диодов до микросхем, аналоговая / цифровая схемотехника.) Мониторы (жидко-кристаллические) - оптика (поляризация). Системы хранения информации: магнитные носители, SSD - ферамагнетики, гистерезис / квантовая механика. \textbf{Преобразование Фурье}. Печатающие устройства - струйные и сублимационные. Квантовые компьютеры.

\section{Электромагнитная теория волноводов}

Уравнение Максвелла для плоской волны:

\begin{remark}
	дифференциальная форма
\end{remark}

\[
\begin{cases}
	\rot \vec{E} = - \dfrac{\partial \vec{B}}{\partial t} \\
	\rot \vec{H} = \vec{j} + \dfrac{\partial \vec{D}}{\partial t}, \vec{j} - \text{ плотность тока} \\
	\divFiz \vec{D} = \rho, \rho - \text{ объёмная плотность заряда} \\
	\vec{B} = 0 \\
	\vec{B} = \mu \mu_o \vec{H} \\
	\vec{D} = \epsilon \epsilon_0 \vec{E}
\end{cases}
\]

\begin{enumerate}
	\item Закон Фарадея. Переменное магнитное поле порождает вихревое электрическое.
	\item Закон полного тока. Переменное электрическое поле порождает вихревое магнитное.
	\item Закон полного тока.
	\item Теорема Гаусса в дифференциальной форме. Источником электростатического поля являются заряды.
	\item Теорема Гаусса для магнитного поля.
\end{enumerate}

$\vec{B}$ - индукции магнитного поля [ТЛ], $\vec{D}$ - индукция электрического поля [$\text{КЛ}/\text{м}^2$], $\vec{H}$ - напряжённость магнитного поля [$\text{А}/\text{м}$], $\vec{E}$ - напряжённость электрического поля [$\text{В}/\text{м}$].

\[
\nabla = 
\begin{bmatrix}
	\dfrac{\partial}{\partial x} \\
	\dfrac{\partial}{\partial y} \\
	\dfrac{\partial}{\partial z} \\
\end{bmatrix}
- \text{ оператор Набла}
\]
\[
\nabla \phi = \vec{i} ~ \frac{\partial \phi}{\partial x} + \vec{j} ~ \frac{\partial \phi}{\partial y} + \vec{k} ~ \frac{\partial \phi}{\partial z} = \grad \phi
\]
\[
\vec{E} = \grad \phi
\]
\[
\nabla \cdot \vec{D} = \divFiz \vec{D} = \frac{\partial Dx}{\partial x} + \frac{\partial Dy}{\partial y} + \frac{\partial Dz}{\partial z}
\]
вектор $\to$ скаляр
\[
\nabla \times \vec{E} = \rot \vec{E} =
\det
\begin{bmatrix}
	\vec{i} & \vec{j} & \vec{k} \\
	\frac{\partial}{\partial x} & \frac{\partial}{\partial y} & \frac{\partial}{\partial z} \\
	E_x & E_y & E_z
\end{bmatrix}
\]
\[
\rot \vec{E} \ne 0 - \text{ вихревое поле}
\]
\[
\vec{E} - \grad \phi - \text{потенциальное поле}
\]

\subsection{Волноводы. Дисперсионные уравнения.}

Основная проблема антенн - волна распространяется во все сторону, направить, конечно, можно, но это очень затратно.

Есть ещё вариант передачи - волноводы. Делятся на СВЧ (10-100 ГГц) и оптические (400-600 нанометров) и периодически-диафрагмированные волноводы.

СВЧ техника не любит, когда в волноводах возникают вакуумные полости или острые углы - могут возникать пробои $\to$ делают скруглёнными, что с аналитической точки зрения - СМЭРТЬ.

\subsubsection{Продольно-однородные волноводы}

Для продольно-однородных волноводах можно доказать, что все компоненты электромагнитного поля выражаются через продольные компоненты, где $E_z ~~~ H_z$.

\[
E_z, E_{\theta}, E_r
\]
\[
H_z, H_{\theta}, H_z
\]

$E_z, H_z$ - продольные компоненты. $E_{\theta}, E_r, H_{\theta}, H_z$ - поперечные компоненты.

\[
\begin{cases}
	E_z = 0 \\
	\frac{\partial H_z}{\partial r} = 0
\end{cases}
\]

Все касательные компоненты непрерывны (первые две):

\[
\begin{cases}
	E_z = const \\
	H_z = const \\
	E_r \epsilon = const \\
	H_r \mu = const
\end{cases}
\]

Нормальные компоненты (вторые две) - разрыв.

\begin{definition}[Фазовая скорость]
	Фазовая скорость - это физическая величина, которая может быть больше скорости света.
\end{definition}

\newpage

\tableofcontents

\end{document} % конец документа